\chapter{Wprowadzenie}
\label{cha:wprowadzenie}

Wraz z rozwojem technologii, ludzie dążą do pełnej automatyzacji w różnych sektorach swojego życia, m.in. przemyśle, bankowości, motoryzacji. Przykładem prężnie rozwijających się rozwiązań pozwalających na pełną lub częściową automatykę są systemy zarządzania budynkami. Są wykorzystywane nie tylko w fabrykach i miejscach pracy, ale zaczynają również pojawiać się w gospodarstwach domowych. Takie systemy, zwane systemami zarządzania budynkiem (ang. \textit{Building Management System}) \cite{BMS}, opierają się na układzie czujników i pozwalają na monitorowanie i zarządzanie wszystkimi urządzeniami w obrębie oraz w bliskim otoczeniu budynku. Pobierają dane na temat wilgotności powietrza, temperatury, a następnie, na podstawie zebranych informacji, sterują podłączonym do systemu sprzętem.\\
Wykorzystanie BMS pozwala na zoptymalizowanie zużycia energii, mediów, poprawienie funkcjonalności, bezpieczeństwa oraz komfortu. \cite{BMS}. Ważne w tym zakresie jest również analizowanie przepływu użytkowników. Ma duży wpływ na ogrzewanie pomieszczeń, potrzebę chłodzenia i wentylacji, energię pochłanianą przez oświetlenie oraz wykorzystanie przestrzeni. Uwzględnianie położenia użytkowników podczas sterowania budynkiem pozwala na zaoszczędzenie jednej-trzeciej zużywanej energii \cite{BECM}.

%---------------------------------------------------------------------------

\section{Cele pracy}
\label{sec:celePracy}

Celem niniejszej pracy jest stworzenie systemu zarządzania budynkiem, który będzie sterował podłączonymi do serwera urządzeniami na podstawie analizy lokalizacji użytkowników. Do kontroli będzie wykorzystywał dane skumulowane, zgromadzone w określony okresie czasu, oraz dane zebrane w czasie rzeczywistym. Lokalizacja użytkowników obliczana będzie na podstawie odczytu siły sygnałów z punktów dostępu WiFi oraz Beaconów Bluetooth, co zmniejszy koszty instalacji systemu, ponieważ nie trzeba będzie kupować dodatkowych, często drogich, czujników. Lokalizacja użytkowników powinna być określana również na podstawie położenia, względem siebie, aplikacji mobilnych, co rozszerzy pulę urządzeń lokalizujących. Jest to szczególnie przydatne w sytuacji, gdy obszar działania systemu będzie posiadał pola, w których sygnały ze stacjonarnych nadajników są zbyt rozproszone, aby w sposób jednoznaczny określić położenie osób korzystających z aplikacji.\\
W swojej pracy będę się starał także zanalizować i wybrać algorytm lokalizacji użytkowników, który będzie najlepiej pasował do wybranego przeze mnie problemu. Podczas wyboru ważną rolę będzie odgrywać odporność algorytmu na błędy pomiarowe oraz odbicia i interferencje sygnału radiowego, na podstawie którego dokonywana będzie lokalizacja.\\
Dodatkowo, będę starał się dokonać analizy dwóch wybranych technologii komunikacji radiowej - WiFi i Bluetooth. Obie technologie będą rozpatrywane pod kątem stabilności siły sygnału w zależności od przeszkód oraz środowiska.\\
Ważne dla całego systemu będzie również stworzenie aplikacji administracyjnej, która pozwoli na konfigurację, monitorowanie położenia oraz analizę przepływu użytkowników.\\