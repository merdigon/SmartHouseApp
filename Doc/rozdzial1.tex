\chapter{Wprowadzenie}
\label{cha:wprowadzenie}

Wraz z rozwojem technologii, ludzie dążą do pełnej automatyzacji w różnych sektorach swojego życia - przemyśle, bankowości, motoryzacji. Rozwinęły się też prężnie systemy automatycznego zarządzania budynkami - inteligentne budynki. Takie systemy są nie tylko wykorzystywane w fabrykach i miejscach pracy, ale również powoli zaczynają pojawiać się w gospodarstwach domowych. Takie systemy, zwane systemami zarządzania budynkiem (Building Management System), operiają się na systemie czujników i pozwalają na monitorowanie i zarządzanie wszystkimi urządzeniami w obrębie oraz w bliskim otoczeniu budynku. Pobierają dane na temat wilgotności powietrza, temperatury, a następnie, na podstawie zebranych danych, sterują podłączonymi do systemu urządzeniami.

%---------------------------------------------------------------------------

\section{Cele pracy}
\label{sec:celePracy}

Celem niniejszej pracy jest stworzenie systemu zarządzania budynkiem, który będzie sterował podłączonymi urządzeniami na podstawie lokalizacji użytkowników. Takie podejście pozwoli na kontrolę nie tylko na podstawie danych odbieranych w danym momencie, ale również zebranych przez pewien okres czasu. Do lokalizacji użytkowników zostaną użyte routery WiFi i Beacony, co zmniejszy koszty instalacji systemu, ponieważ nie trzeba będzie kupować dodatkowych, często drogich, czujników. Dodatkowo, lokalizacja użytkowników będzie określana na podstawie położenia ich samych względem siebie, co rozszerza pulę urządzeń lokalizujących. Jest szczególnie przydatne w sytuacji, gdy obszar działania systemu będzie posiadał pola, w których sygnały ze stacjonarnych nadajników są zbyt rozproszone, aby w sposób jednoznaczny określić położenie użytkowników.\\
W swojej pracy będę się starał także zanalizować i wybrać algorytm lokalizacji użytkowników, który będzie najlepiej pasował do wybranego przeze mnie problemu. Podczas wyboru ważną rolę będzie odgrywać odporność algorytmu na błędy pomiarowe oraz odbicia i interferencje sygnału radiowego, na podstawie którego dokonywana będzie lokalizacja.\\
Dodatkowo, będę starał się dokonać analizy dwóch wybranych technologii komunikacji radiowej - WiFi i Bluetooth. Obie technologie będą rozpatrywane pod kątem stabilności siły sygnału w zależności od przeszkód oraz ukierunkowania transmitera.\\
Ważne dla całego systemu będzie również stworzenie aplikacji administracyjnej, która pozwoli na konfigurację, monitorowanie położenia oraz analizę przepływu użytkowników.\\