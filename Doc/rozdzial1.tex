\chapter{Wprowadzenie}
\label{cha:wprowadzenie}

Wraz z rozwojem technologii, ludzie dążą do pełnej automatyzacji w różnych sektorach swojego życia - przemyśle, bankowości, motoryzacji. Z tego powodu, prężnie rozwinęły się systemy automatycznego zarządzania budynkami. Są wykorzystywane nie tylko w fabrykach i miejscach pracy, ale również powoli zaczynają pojawiać się w gospodarstwach domowych. Takie systemy, zwane systemami zarządzania budynkiem (Building Management System) \cite{BMS}, opierają się na układzie czujników i pozwalają na monitorowanie i zarządzanie wszystkimi urządzeniami w obrębie oraz w bliskim otoczeniu budynku. Pobierają dane na temat wilgotności powietrza, temperatury, a następnie, na podstawie zebranych informacji, sterują podłączonym do systemu sprzętem.

%---------------------------------------------------------------------------

\section{Cele pracy}
\label{sec:celePracy}

Celem niniejszej pracy jest stworzenie systemu zarządzania budynkiem, który będzie sterował podłączonymi urządzeniami na podstawie lokalizacji użytkowników. Takie podejście pozwoli na kontrolę nie tylko na podstawie danych odbieranych w danym momencie, ale również zebranych przez pewien okres czasu. Do lokalizacji użytkowników zostaną użyte routery WiFi i Beacony, co zmniejszy koszty instalacji systemu, ponieważ nie trzeba będzie kupować dodatkowych, często drogich, czujników. Lokalizacja użytkowników powinna być określana również na podstawie położenia, względem siebie, aplikacji mobilnych, co rozszerzy pulę urządzeń lokalizujących. Jest to szczególnie przydatne w sytuacji, gdy obszar działania systemu będzie posiadał pola, w których sygnały ze stacjonarnych nadajników są zbyt rozproszone, aby w sposób jednoznaczny określić położenie osób korzystających z aplikacji.\\
W swojej pracy będę się starał także zanalizować i wybrać algorytm lokalizacji użytkowników, który będzie najlepiej pasował do wybranego przeze mnie problemu. Podczas wyboru ważną rolę będzie odgrywać odporność algorytmu na błędy pomiarowe oraz odbicia i interferencje sygnału radiowego, na podstawie którego dokonywana będzie lokalizacja.\\
Dodatkowo, będę starał się dokonać analizy dwóch wybranych technologii komunikacji radiowej - WiFi i Bluetooth. Obie technologie będą rozpatrywane pod kątem stabilności siły sygnału w zależności od przeszkód oraz środowiska.\\
Ważne dla całego systemu będzie również stworzenie aplikacji administracyjnej, która pozwoli na konfigurację, monitorowanie położenia oraz analizę przepływu użytkowników.\\