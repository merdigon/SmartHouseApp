\chapter{Propagacja sygnału radiowego}
\label{cha:teoria}
\section{Zanik sygnał radiowego}
Sygnał radiowy, wysłany przez urządzenie, w wyniku przechodzenia przez ośrodek (najczęściej jest nim powietrze) ulega zanikowi. \cite{ST} Różnica w mocy sygnału odczytanego przez odbiornik, a siłą sygnału wysłanego przez nadawcę, pozwala na obliczenie odległości pomiędzy dwoma urządzeniami. Dodatkowymi współczynnikami, które mają wpływ na obliczenia, są:
\begin{itemize}
	\item zysk energetyczny anteny nadawcy oraz odbiorcy
	\item margines zaniku, który zależy od czułości odbiornika
	\item straty w sile sygnału wynikające ze środowiska: przeszkody, odbicia, inne urządzenia nadające
\end{itemize}
\begin{figure}[H]			
	\centering
	\caption{Schemat zaniku sygnału}
	\includegraphics[width=0.75\textwidth]{zanik_sygnalu}
\end{figure}
\section{Zysk energetyczny anteny}
Zysk energetyczny anteny jest to stosunek mocy anteny wypromieniowanej w danym kierunku do mocy wypromieniowanej przez antenę wzorcową \cite{ML}. Anteną wzorcową może być m.in. antena izotropowa, czyli antena bez fizycznych rozmiarów, która cały sygnał zasilany wysyła we wszystkich kierunkach. W takim wypadku, zysk energetyczny anteny wyrażany jest w $dBi$ \cite{AG}.\\
Na zysk energetyczny mają również wpływ kierunkowość oraz materiał, z którego wykonana jest antena.
\section{Margines zaniku}
Margines zaniku (ang. \textit{fade margin}) jest to wartość określająca różnicę pomiędzy wartością siły odebranego sygnału, a czułością odbiornika \cite{RFM}. Określa się ją wzorem:
\begin{equation}
FM = P_{rs} - P_{rx}
\end{equation}
gdzie $P_{rs}$ to siła odebranego sygnału.\\
Margines zaniku jest wyznacznikiem, jak "dobre" jest połączenie między transmiterem i odbiorcą \cite{FMC}. Czym wyższa wartość marginesu zaniku, tym bardziej niezawodne jest połączenie.
Wartość marginesu zaniku określa się w jednostce $dB$.
\section{Received Signal Strength Indication}
\textit{Received Signal Strength Indication} (skrótem RSSI) jest to miara określająca moc sygnału odbieranego. Przyjmuje ona wartości niedodatnie (gdzie 0 oznacza sygnał najsilniejszy). Jednostką, w jakiej określa się siłę sygnału, jest $dBm$, która jest logarytmiczną jednostką miary mocy odniesioną do mocy $1mW$ \cite{AMMC}.\\
System Android pozwala na odczytanie siły odbieranego sygnału. Można do tego wykorzystać API $WifiManager$ (w przypadku odczytu sygnału WiFi) oraz $BroadcastReceiver$ (w przypadku odczytu sygnału Bluetooth) \cite{ADBR}.
\section{Free-space path loss}		  		  
\textit{Free-space path loss} (FSPL) jest to utrata siły sygnału, spowodowana przejściem fali elektromagnetycznej przez ośrodek (najczęściej powietrze) \cite{SPPLM}.\\
Wzór na obliczanie FSPL \cite{TPL}:
\begin{equation}
FSPL = P_{tx} + AG_{tx} + AG_{rx} - P_{rx} - FM - L
\end{equation}
%http://electronicdesign.com/communications/understanding-wireless-range-calculations
%Microwave and Millimetre-Wave Design for Wireless Communications  Autorzy Ian Robertson,Nutapong Somjit,Mitchai str 449
%https://en.wikipedia.org/wiki/Free-space_path_loss
%http://www.tplink.com/ie/support/calculator/#1
%http://stackoverflow.com/questions/11217674/how-to-calculate-distance-from-wifi-router-using-signal-strength
Gdzie symbole oznaczają:
\begin{itemize}
	\item $P_{tx}$ - siła transmitera, wyrażona w dBm
	\item $AG_{tx}$ - zysk energetyczny anteny transmitera, wyrażony w dBi
	\item $AG_{rx}$ - zysk energetyczny anteny odbiorcy, wyrażony w dBi
	\item $P_{rx}$ - czułość odbiornika, wyrażona w dBm
	\item $FM$ - margines zaniku sygnału (fade margin) - określa jak duży jest margines różnicy pomiędzy uzyskaną siłą sygnału, a czułością odbiornika
	\item $L$ - straty wynikające np z oddziaływania innych transmiterów, przeszkód itp.
\end{itemize}
Dodatkowo, FSPL można obliczyć, używając następujący wzór \cite{EN}:
\begin{equation}
FSPL = 20log_{10}\left(\frac{d}{d_{0}}\right) + 20log_{10}(f) + K
\end{equation}
Gdzie symbole oznaczają:
\begin{itemize}
	\item $d$ - dystans dzielący transmiter od odbiorcy, wyrażony w metrach
	\item $d_{0}$ - dystans referencyjny -  w tym wypadku 1 metr
	\item $f$ - częstotliwość transmitera - wyrażona w MHz
	\item $K$ - stała, którą można określić wzorem:
	\begin{equation}
	K = 20log_{10}\left(\frac{4\pi d_{0}}{C}\right)
	\end{equation}
	gdzie $d_{0}$ to dystans referencyjny (taki sam jak we wzorze wyżej), a $C$ to długość fali emitowanej przez transmiter
\end{itemize}	  
Po przekształceniu wzoru, uzyskujemy:
\begin{equation}
d = 10^{\left(\frac{FSPL - K - 20log_{10}(f)}{20}\right)}
\end{equation}
A po połączeniu obu wzorów dostajemy:
\begin{equation}
d = 10^{\left(\frac{P_{tx} + AG_{tx} + AG_{rx} - P_{rx} - FM - L - K - 20log_{10}(f)}{20}\right)}
\end{equation}
Po podstawieniu wzoru na margines zaniku do wzoru (3.6), uzyskujemy
\begin{equation}
d = 10^{\left(\frac{P_{tx} + AG_{tx} + AG_{rx} - P_{rs} - L - K - 20log_{10}(f)}{20}\right)}
\end{equation}
\section{Wpływ współczynników sygnału na określanie lokalizacji}
Przedstawione powyżej współczynniki mają duży wpływ na dokładność i jakość obliczonej pozycji użytkownika. Najważniejszym współczynnikiem są straty, które należy uwzględnić podczas wyznaczania odległości. Czym więcej przeszkód i zakłóceń spotka na swojej drodze sygnał, tym jego siła będzie mniejsza. Wiąże się z tym duży błąd podczas wyznaczania lokalizacji odbiornika \cite{RSC}.\\
Dodatkowy wpływ na obliczenia ma siła obu anten - transmitera i odbiornika. Im większy jest ich zysk energetyczny, tym sygnał jest bardziej odporny na straty, a co za tym idzie, dystans pomiędzy dwoma urządzeniami, obliczony na podstawie \textit{FSPL}, jest dokładniejszy i lepiej odzwierciedla rzeczywiste położenie \cite{FSPL}.\\
Również margines zaniku ma duży wpływ na jakość odbieranego sygnału - im jest wyższy, tym rzadziej występuje minimum poziomu wydajności, które negatywnie wpływa na stabilność siły sygnału \cite{MT}.
