\documentclass{article}
\usepackage{polski}
\usepackage{textcomp}
\usepackage{amssymb}
\usepackage[utf8]{inputenc}
\date{2016-10-31}
\usepackage{graphicx}
\author{Szymon Nowak}
\begin{document}
  \section{Teoria}
	  \subsection{Free-space path loss}		  		  
		  Utrata w sile sygnału spowodowana przejściem fali elektromagnetycznej przez ośrodek (najczęściej powietrze).
		  Wzór na obliczanie FSPL:
		  \begin{equation}
			  FSPL = P_{tx} + AG_{tx} + AG_{rx} - P_{rx} - FM - L
		  \end{equation}
		  %http://electronicdesign.com/communications/understanding-wireless-range-calculations
		  %Microwave and Millimetre-Wave Design for Wireless Communications  Autorzy Ian Robertson,Nutapong Somjit,Mitchai str 449
		  %https://en.wikipedia.org/wiki/Free-space_path_loss
		  %http://www.tplink.com/ie/support/calculator/#1
		  %http://stackoverflow.com/questions/11217674/how-to-calculate-distance-from-wifi-router-using-signal-strength
		  Gdzie symbole oznaczają:
		  \begin{itemize}
		  	\item $P_{tx}$ - siła trasmitera, wyrażona w dBm
		  	\item $AG_{tx}$ - zysk energentyczny anteny transmitera, wyrażony w dBi
		  	\item $AG_{rx}$ - zysk energentyczny anteny odbiorcy, wyrażony w dBi
		  	\item $P_{rx}$ - siła odbiornika, wyrażona w dBm
		  	\item $FM$ - margines zaniku sygnału (fade margin)
		  	\item $L$ - straty wynikające np z oddziaływania innych transmiterów, przeszkód itp.
		  \end{itemize}
		  Dodatkowo, FSPL można obliczyć, używając następujący wzór:
		  \begin{equation}
			  FSPL = 20log_{10}\left(\frac{d}{d_{0}}\right) + 20log_{10}(f) + K
		  \end{equation}
		  %Indoor Localization Method Based on Wi-Fi Trilateration Technique Maxim Shchekotov
		  % Wi-Fi Indoor Positioning System Based on RSSI Measurements from Wi-Fi  Access Points  –  A Tri-lateration Approach Onkar Pathak, Pratik Palaskar, Rajesh Palkar, Mayur Tawari
		  Gdzie symbole oznaczają:
		  \begin{itemize}
		  	\item $d$ - dystans dzielący trasmiter od odbiorcy, wyrażony w metrach
		  	\item $d_{0}$ - dystans referencyjny -  w tym wypadku 1 metr
		  	\item $f$ - częstotliwość transmitera - wyrażona w MHz
		  	\item $K$ - stała, którą można określić wzorem:
			  	\begin{equation}
				  	K = 20log_{10}\left(\frac{4\pi d_{0}}{C}\right)
			  	\end{equation}
			  	gdzie $d_{0}$ to dystans referencyjny (taki sam jak we wzorze wyżej), a $C$ to długość fali emitowanej przez transmiter
		  \end{itemize}
	  
		  Po przekształceniu wzoru, uzytkujemy:
		  \begin{equation}
			  d = 10^{\left(\frac{FSPL - K - 20log_{10}(f)}{20}\right)}
		  \end{equation}
		  A po połączeniu obu wzorów dostajemy:
		  \begin{equation}
		  d = 10^{\left(\frac{P_{tx} + AG_{tx} + AG_{rx} - P_{rx} - FM - L - K - 20log_{10}(f)}{20}\right)}
		  \end{equation}
		\subsection{Zysk energetyczny anteny}
			Zysk energetyczny anteny jest to stosunek mocy ateny wypromieniowanej w danym kierunku do mocy wypromieniowanej przez antenę wzorcową. Anteną wzorcową może być m.in. antena izotropowa, czyli antena bez fizycznych rozmiarów, która cały sygnał zasilany wysyła we wszystkich kierunkach. W takim wypadku, zysk energetyczny anteny wyrażany jest w $dBi$.\\
			Na zysk energetyczny mają również wpływ kierunkowość oraz materiał, z którego wykonana jest antena.
		\subsection{Received signal strength indication}
			Received signal strength indication (skrótem RSSI) jest to miara określająca moc sygnału odbieranego. Przyjmuje ona wartości niedodatnie (gdzie 0 oznacza sygnał najsilniejszy). Jednostką, w jakiej określa się siłę sygnału jest $dBm$, która jest logarytmiczną jednostką miary mocy odniesiona do mocy $1mW$.\\
			System Android pozwala na odczytanie siły odbieranego sygnału. Można do tego wykorzystać API $WifiManager$ (w przypadku odczytu sygnału WiFi) oraz $BroadcastReceiver$ (w przypadku odczytu sygnału Bluetooth).
	\section{Bluetooth}
		%http://bluetoothinsight.blogspot.com/2008/01/bluetooth-power-classes.html
	\section{Eksperymenty}
		\subsection{Wykorzystane urządzenia}
			\begin{enumerate}
				\item Smartphone Sony Xperia Z1 Compact (D5503) - odbiornik\\				
					Dane techniczne:
					\begin{itemize}
						\item Częstotliwość - 2,4GHz
						\item Przyrost siły sygnału z anteny - 2dBi
					\end{itemize}
				\item Router TP-Link TD-W8970 - nadajnik\\
				Dane techniczne:
				\begin{itemize}
					\item Częstotliwość - 2,4GHz
					\item Dwie zewnętrzne anteny kierunkowe
					\item Przyrost siły sygnału z anteny - 4dBi
					\item Siła transmitera - 16.5dBm					
				\end{itemize}
				\item Router TP-Link TL-WA701ND - nadajnik\\
				Dane techniczne:
				\begin{itemize}
					\item Częstotliwość - 2,4GHz
					\item Jedna zewnętrzna antena kierunkowa
					\item Przyrost siły sygnału z anteny - 2dBi
					\item Siła transmitera - 15dBm					
				\end{itemize}
				\item Smartphone Grand 2 (G7102) - nadajnik\\
				Dane techniczne:
				\begin{itemize}
					\item Częstotliwość - 2,4GHz
					\item Jedna antena wbudowana
					\item Przyrost siły sygnału z anteny - 0dBi
					\item Siła transmitera - 10dBm					
				\end{itemize}
			\end{enumerate}
		\subsection{Warunki}
			Wszystkie pomiary wykonywane były w pomieszczeniu zamknięty, bez przeszkód na drodze sygnału, dlatego jako margines zaniku sygnału została przyjęte wartość 22 dBm. Inne straty (np interferencja sygnałów z routerów) zostały pominięte i ich wykrycie jest jednym z celów eksperymentu.
		\subsection{Cele}
			Celem eksperymentu jest ustalenie, jak zmierzona i obliczona, przy użyciu siły sygnałów, odległość między odbiornikiem i transmiterami odnosi się do odległości rzeczywistej. Dodatkowo, będę się starał ustalić, jak duży wpływ na jakość sygnału mają przeszkody, kierunek, w jakim skierowane są względem siebie urządzenia oraz interferencja sygnałów.
		\subsection{Pomiar odległości}
			Eksperyment polegał na ustawieniu transmitera 1m od odbiornika na jednym poziomie, antenami do siebie. Następnie dodawana była przeszkoda (w tym wypadku książka) i pomiary zostały powtórzone. Eksperyment został wykonany dla wszystkich transmiterów.\\			
			\begin{figure}	
				\centering			
				\caption{Szkic eksperymentu nr 1}
				\includegraphics{exper1}
			\end{figure}
			\begin{itemize}
				\item Router TP-Link TD-W8970
				\begin{center}
					\begin{minipage}{\linewidth}
					Wersja bez przeszkody:\\\\
					\begin{tabular}{|c|c|c|}
						\hline 
						Pomiar & Siła sygnału (w dBm) & Obliczona odległość (w metrach) \\ 
						\hline 
						1 & -41 & 1,16 \\ 
						\hline 
						2 & -40 & 1,04 \\ 
						\hline 
						3 & -37 & 0.73 \\ 
						\hline 
						4 & -42 & 1.30 \\ 
						\hline 
						5 & -37 & 0,73 \\ 
						\hline 
					\end{tabular} 
				\end{minipage} 
				\end{center}
				\begin{center}
					\begin{minipage}{\linewidth}
						Wersja z przeszkodą:\\\\
					\begin{tabular}{|c|c|c|}
						\hline 
						Pomiar & Siła sygnału (w dBm) & Obliczona odległość (w metrach) \\ 
						\hline 
						1 & -38 & 0,82 \\ 
						\hline 
						2 & -39 & 0,92 \\ 
						\hline 
						3 & -42 & 1,30 \\ 
						\hline 
						4 & -46 & 2,07 \\ 
						\hline 
						7 & -43 & 1,46 \\ 
						\hline 
					\end{tabular}				
					\end{minipage} 
				\end{center}
			\item Router TP-Link TL-WA701ND
			\begin{center}
				\begin{minipage}{\linewidth}
					Wersja bez przeszkody:\\\\
					\begin{tabular}{|c|c|c|}
						\hline 
						Pomiar & Siła sygnału (w dBm) & Obliczona odległość (w metrach) \\ 
						\hline 
						1 & -44 & 0,98 \\ 
						\hline 
						2 & -44 & 0,98 \\ 
						\hline 
						3 & -45 & 1,10 \\ 
						\hline 
						4 & -47 & 1,38 \\ 
						\hline 
						5 & -45 & 1,10 \\ 
						\hline 
					\end{tabular} 
				\end{minipage} 
			\end{center}
			\begin{center}
				\begin{minipage}{\linewidth}
					Wersja z przeszkodą:\\\\
					\begin{tabular}{|c|c|c|}
						\hline 
						Pomiar & Siła sygnału (w dBm) & Obliczona odległość (w metrach) \\ 
						\hline 
						1 & -49 & 1,74 \\ 
						\hline 
						2 & -47 & 1,38 \\ 
						\hline 
						3 & -46 & 1,23 \\ 
						\hline 
						4 & -47 & 1,38 \\ 
						\hline 
						5 & -47 & 1,38 \\ 
						\hline 
					\end{tabular}//
				\end{minipage} 
			\end{center}
		\item Samsung Grand 2
		\begin{center}
			\begin{minipage}{\linewidth}
				Wersja bez przeszkody:\\\\
				\begin{tabular}{|c|c|c|}
					\hline 
					Pomiar & Siła sygnału (w dBm) & Obliczona odległość (w metrach) \\ 
					\hline 
					1 & -51 & 1,38 \\ 
					\hline 
					2 & -50 & 1,23 \\ 
					\hline 
					3 & -49 & 1,10 \\ 
					\hline 
					4 & -48 & 0,98 \\ 
					\hline 
					5 & -53 & 1,74 \\ 
					\hline 
				\end{tabular} 
			\end{minipage} 
		\end{center}
		\begin{center}
			\begin{minipage}{\linewidth}
				Wersja z przeszkodą:\\\\
				\begin{tabular}{|c|c|c|}
					\hline 
					Pomiar & Siła sygnału (w dBm) & Obliczona odległość (w metrach) \\ 
					\hline 
					1 & -50 & 1,23 \\ 
					\hline 
					2 & -54 & 1,95 \\ 
					\hline 
					3 & -53 & 1,74 \\ 
					\hline 
					4 & -55 & 2,19 \\ 
					\hline 
					5 & -55 & 2,19 \\ 
					\hline 
				\end{tabular}//
			\end{minipage} 
		\end{center}
		\end{itemize}
		
		\subsection{Pomiary zakłóceń}
		Eksperyment polegał na rozmieszczeniu trasmiterów na wierzchołkach trójkąta, w środku którego znajdował się odbiornik. Wszystkie urządzenia znajdowały się na tej samej wysokości. Mierzone były zmiany siły sygnału i obliczonej odległości w zależności od kąta położenia odbiornika w stosunku do trasmitera oraz ilości nakładających się na siebie sygnałów. Na początku, włączony był tylko transmiter o indeksie A. Odbiornik znajdował się w stosunku do transmitera pod kątem około 50 stopni. Następnie włączony został transmiter B. Na końcu do modelu został dodany trasmiter C.\\
		
		\begin{figure}				
			\centering
			\caption{Model systemu do pomiaru zakłóceń}
			\includegraphics{inz2}
		\end{figure}
		Informacje o urządzeniach:
		\begin{itemize}
			\item Transmiter A - TP-Link TD-W8970, współrzędne (1.80, 0)
			\item Transmiter B - TP-Link TL-WA701ND, współrzędne (0, 0)
			\item Transmiter C - Samsung Grand 2, współrzędne (1.07, 1.8)
			\item Odbiornik - współrzędne (1.2, 0.45)
		\end{itemize}
		\begin{center}
			\begin{minipage}{\linewidth}
				Pomiar bez zakłóceń dla odległości 80cm przy kącie 50\textdegree :\\\\
				\begin{tabular}{|c|c|c|}
					\hline 
					Pomiar & Siła sygnału (w dBm) & Obliczona odległość (w metrach) \\ 
					\hline 
					1 & -41 & 1,16 \\ 
					\hline 
					2 & -43 & 1,46 \\ 
					\hline 
					3 & -41 & 1,16 \\ 
					\hline 
					4 & -42 & 1,30 \\ 
					\hline 
					5 & -43 & 1,46 \\ 
					\hline 
				\end{tabular}
			\end{minipage} 
		\end{center}
		\begin{center}
			\begin{minipage}{\linewidth}
				Pomiar z zakłóceniami z transmitera B dla odległości 80cm przy kącie 50\textdegree :\\\\
				\begin{tabular}{|c|c|c|}
					\hline 
					Pomiar & Siła sygnału (w dBm) & Obliczona odległość (w metrach) \\ 
					\hline 
					1 & -44 & 1,64 \\ 
					\hline 
					2 & -47 & 2,31 \\ 
					\hline 
					3 & -45 & 1,84 \\ 
					\hline 
					4 & -48 & 2,60 \\ 
					\hline 
					5 & -47 & 2,31 \\ 
					\hline 
				\end{tabular}
			\end{minipage} 
		\end{center}
		\begin{center}
			\begin{minipage}{\linewidth}
				Pomiar z zakłóceniami z obu transmiterów dla odległości 80cm przy kącie 50\textdegree :\\\\
				\begin{tabular}{|c|c|c|}
					\hline 
					Pomiar & Siła sygnału (w dBm) & Obliczona odległość (w metrach) \\ 
					\hline 
					1 & -48 & 2,60 \\ 
					\hline 
					2 & -47 & 2,31 \\ 
					\hline 
					3 & -44 & 1,64 \\ 
					\hline 
					4 & -48 & 2,60 \\ 
					\hline 
					5 & -46 & 2,06 \\ 
					\hline 
				\end{tabular}
			\end{minipage} 
		\end{center}
	\subsection{Wyznaczanie lokalizacji użytkownika}
	  Narazie mało do napisania. Z dwóch pomiarów dla modelu z góry, dostałem lokalizacje (-0,4; 1,2; -0,3) oraz (1,5; 1,67; -1,5).
\section{Model wyznaczania lokalizacji}
	Stworzyłem model algorytmu w MatLabie. Nie wyobrażam sobie modelu w trzech wymiarach i z kolorem (według mnie wynikiem będzie prostopadłościan o granatowym dominującym kolorze ścian) , dlatego stworzyłem model 2D, który jest tak naprawdę przekrojem modelu 3D (płaszczyzną XY).
	\begin{figure}	
		\centering			
		\caption{Model systemu z trzema routerami}
		\includegraphics[width=\textwidth]{guasianRouter}
	\end{figure}
	Algorytm zmieniłem według zastrzeżeń Pana Doktora. Prostopadłościan, który zawiera w sobie "sfery" ruterów, dzielony jest na max 20 kawałków. Wyznacza się najlepszą pozycję, dla niej brane są sąsiadujące pozycję i uzyskany sześcian dzieli się na 9 kawałków i ponownie wyznacza najlepszą pozycję. Obliczenia kończą się, jak spełnione jest założenie: $szerKawalka \leqslant okreslonaDokladnosc$.
	
\section{Implementacja}
	\subsection{Serwer}
		Serwer jest aplikacją, która ma analizować zabrane dane, sterować urządzeniami zewnętrznymi oraz stanowić most pomiędzy klientem administracyjnym, a aplikacjami mobilnymi. Na zadania aplikacji serwerowej składają się:
		\begin{itemize}
			\item Zbieranie danych z aplikacji mobilnych oraz wyznaczanie lokalizacji użytkowników
			\item Zezwalanie użytkownikowi administracyjnemu na konfigurację systemu poprzez żądań aplikacji klienckiej
			\item Sterowanie i zarządzanie urządzeniami zewnętrznymi analizując dane odebrane z aplikacji mobilnych
			\item Dostarczanie, w formie skumulowanej lub w czasie rzeczywistym, danych na temat położenia użytkowników do analizy i monitoringu
		\end{itemize}
		\subsubsection{Lokalizacja użytkowników mobilnych}
		\paragraph{Pobranie i analiza danych}
		Serwer pobiera dane od użytkowników w formie requestów HTTP. Każde żądanie wysłane do serwera musi posiadać adres MAC urządzenia Bluetooth wysyłającego oraz listę zarejestrowanych sygnałów. Każda pozycja na liście sygnałów musi posiadać nazwę urządzenia, którego sygnał został odebrany (w przypadku sygnału wysłanego przez router jest to SSID sieci, zaś w przypadku urządzeń Bluetooth jest to adres MAC), typ sygnału (WIFI albo Bluetooth) oraz zarejestrowaną siłę sygnału, określoną w dBm. \\
		Następnie, dla każdego elementu z listy dociągane są stałe dane zarejestrowane w systemi - lokalizacja oraz waga sygnału. Dane na temat routerów oraz stałych urządzeń Bluetooth (np. Beconów) pobierane są bazy danych. Jeżeli jakiś sygnał Bluetooth nie widnieje w bazie danych, sprawdzane są ostatnie żądania od urządzeń mobilnych, dla których udało się określić lokalizacjęa, a czas od ostatniej aktualizacji nie jest większy niż 4 sekundy. Jeżeli sygnał pochodzi od jednego z tych urządzeń, potrzebne informacje pobierane są z dynamicznie budowanej, lokalnej bazy wiedzy. Jeżeli sygnał nie figuruje ani w bazie danych, ani w bazie dynamicznej, zostaje uznany jako sygnał przypadkowy i odrzucony. Waga sygnału przyjmuje wartości w skali od 1 do 4. Domyślnie, sygnałowi pochodzącemu od routera WiFi nadawana jest waga 3, sygnałowi ze stałego urządzenia Bluetooth waga 2, zaś sygnały pochodzące od innych użytkowników mobilnych wagę 1. Wagę stałych urządzeń WiFi i Bluetooth można edytowac przy użyciu panelu konfiguracyjnego w aplikacji klienckiej. W ostatnim kroku, dla każdego zarejestrowanego sygnału obliczana jest odległość urządzenia od użytkownika. Wykorzystywana do tego jest lokalizacja użytkownika, wzór na Free-space path loss oraz dane statyczne (jak siła anten, siła trasmitera itp.).\\
		\paragraph{Algorytmiczne wyznaczenie lokalizacji}
		Celem algorytmu jest wyznaczenie punktu, dla którego suma prawdopodobieństw wynikających z odległości użytkownika od urządzenia, jest największa.\\
		Dane pobrane od użytkownika, uzupełnione o statyczne dane przechowywane w systemie, przekazane są do sekcji napisanej w języku F\#. Na wstępie, do każdego zarejestrownego algorytmu zostaje przypisana probabilistyczna Guassa, określająca prawdopodobieństwo znalezienia się użytkownika w danym punkcie w przestrzeni, gdzie stała $\mu$ przyjmuje wartość równą dystansowi obliczonemu na podstawie siły sygnału urządzenia. Dzięki takiemu podejściu, każdy sygnał można zwizualizować jako sferę, której powierzchnia zbliżona jest do chmury. Największe zagęszczenie prawdopodobieństwa występuję dla średnicy równej odległości obliczonej z siły sygnału, a która rzednie zbliżając się i oddalając od środka sfery.\\
		Pierwszym krokiem algorytmu jest wyznaczenie prostapadłościanu, dla którego wykonywane będą obliczenia. Wielkość bryły dobrana jest tak, aby wewnątrz niej znalazły się wszystkie sfery sygnałów (przy uwzględniu zapasu równego 2$\sigma$). Następnie prostopadłościan oraz sfery są normalizowane w taki sposób, aby początek układu zaczył się w punkcie (0,0,0), zaś wszystkie wartości współrzędnych przyjmowały tylko wartości nieujemne. Celem takiej operacji jest uproszczenie algorytmu oraz wyzbycie się potrzeby skalowania iteratorów oraz odnośników do elementów w tablicach.\\
		Kolejnym krokiem algorytmu jest podział prostopadłościanu na części, dla których liczona będzie suma prawdopodobieństw. Celem tego kroku jest podział pola działania na jednakowe sześciany w takie sposób, aby w żadnym wymiarze ilość sześcianów nie przekroczyła 100. Aby to uzyskać, najdłuższy bok prostopadłościanu zostaje podzielony na 100 częście. Następnie boki w pozostałych 2 wymiarach zostają podzielone na sześciany o krawędziach równej długości. Dzięki takiemu zabiegowi dzieli się prostopadłościan na sześciany. Podział prostopadłościanu na sześciany pozwala na wyeliminowanie błędów obliczeniowych wynikających z nierealistycznego podziału pola oliczeń.\\
		Kolejnym krokiem algorymtu jest wyliczenie sumy prawdopodobieńst ze wszystkich sfer sygnałów dla każdego sześcianu w prostopadłościanie. Każde prawdopodieństo, będące składową sumy, przemnażane jest przez wagę danego sygnału. Następnie wybierany jest sześcian, dla którego suma prawdopodobieństwa jest najwyższa. Jeżeli długość boku sześcianu jest równa lub mniejsza od naszego przybliżenia, współrzędna sześcianu staje się lokalizacją naszego użytkownika. Jeżeli długość boku sześcianu jest większa od naszego przybliżenia, dla wybranego sześcianu dobierane są jego sześciany sąsiednie. Następnie wybrane 27 sześcianów staje się nowym modelem obliczeniowym. Każdy wymiar nowego pola dzialone jest na 9 równych części, dzięki czemu uzyskuje się 729 sześcianów. Algorytm zostaje powtórzony, aż lokalizacja nie zostanie określona z interesującym nas przybliżeniem. Sześcian o największej sumie prawdopodobieństw staje się lokalizacją użytkownika.\\
		Ostatnim krokiem algorytmu jest przeliczenie obliczonej lokalizacji przy użyciu danych uzyskanych podczas normalizacji, aby obliczona lokalizacja odpowiadała lokalizacji dla danych wejściowych.
		\paragraph{Zarządzanie lokalizacją użytkownika}
		Po pozytywnym obliczeniu lokalizacji, zostaje ona zapisana w bazie danych, aby potem mogła być użyta do wyświetlenia skumulowanej mapy przepływu użytkowników albo do sterowania urządzeniami. Następnie, lokalizacja zostaje asynchronicznie wysłana do wszystkich klientów administracyjnych, którzy zarejestrowali swoją chęć pobierania danych w trybie real-time (w czasie rzeczywistym). Następnie, lokalizacja wraz z adresem MAC użytkownika zostaje przekazana do wątków urządzeń zewnętrznych, które wykorzystują te dane do podjęcia decyzji o wywołaniu przypisanego urządzeniowi eventu. W ostatnim kroku, lokalizacja zostaje dodana (lub podmieniona, jeżeli wpis o danym użytkowniku już istnieje) w bazie dynamicznej, aby ta informacja mogła posłużyć przy wyznaczaniu lokalizacji innych użytkowników.
		\subsubsection{Sterowanie i zarządzanie urządzeniami}
		Serwer, poza pobieraniem i analizą danych, zajmuje się również sterowaniem przydzielonymi mu urządzeniami. Informacje na temat urządzeń przechowywane są w bazie danych. Danymi, które są potrzebne do sterowania urządzeniem, niezależnie od jego typu są:
		\begin{itemize}
			\item Jego lokalizacja (określona przez 3 współrzędne)
			\item Ip urządzenia oraz port, na którym nasłuchuje
			\item Nazwę rozpoznawalną przez użytkownika (np żarówka na korytarzu)
			\item Sterownik określający sposób komunikacji, implementujący interfejs przypisany do konkretnego typu urządzenia
			\item Moduł określający, czy dla danego urządzenia przypisane jest jakieś sterowanie eventami (np natychmiastowa zmiana siły oświetlenia, spowodowana zbliżeniem się określonego użytkownika)
			\item Flaga określająca, czy dane urządzenie jest aktywne
		\end{itemize}
		Dla każdego aktywnego urządzenia zapisanego w systemie uruchomiony jest na serwerze osobny wątek sterujący. Taki sposób pracy został przyjęty, aby sterowanie i obsługa eventów odbywała się płynnie i w równy sposób dla wszystkich urządzeń, a błąd czy problemy komunikacyjne jednego z urządzeń nie miały wpływu na inne. Do każdego wątku sterującego przypisany jest obiekt klasy zawierającej wszystkie potrzebne informacje oraz metody, aby kontrolować dany typ urządzenia (dla oświetlenia jest to klasa LightDeviceControllingThread)
\end{document}











