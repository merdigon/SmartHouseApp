\chapter{Testy systemu}
\section{Lokalizowanie użytkownika}
\section{Monitorowanie i konfiguracja systemu}
W celu przetestowania w działania aplikacji klienckiej, została utworzona osobna aplikacja, której celem było symulowanie aplikacji mobilnej. Aplikacja testująca składała się z panelu, którego rozmiar był dostosowany do wielkości obszaru pracy systemu. W momencie, jak użytkownik kliknął wewnątrz panelu, pozycja myszki została przeliczana na rzeczywistą pozycję w systemie. Następnie aplikacja, przy użyciu danych na temat statycznych transmiterów, obliczała dystans, a następnie siłę sygnału do każdego z zarejestrowanych routerów i Beaconów. Tak obliczone dane wysyłane były na serwer protokołem HTTP, na adres odpowiedniego kontrolera. Tak sformatowane zgłoszenie było rozpatrywane przez serwer tak samo, jakby informacje były wysłane przez aplikację mobilną.\\
\section{Podsumowanie}
Celem pracy było stworzenie systemu, który określa położenie użytkownika w trzech wymiarach na podstawie siły sygnału znanych ruterów, stałych urządzeń Bluetooth (np Beacony) oraz dynamicznie zmieniających się sił sygnałów Bluetooth innych użytkowników systemu. Obliczone lokalizacje miały być przechowywane oraz przedstawienie ich na mapie budynku w sposób łatwy do analizy. Dodatkowo, celem było opracowanie algorytmu, który na podstawie określonych warunków oraz mapy przepływu zarządza wybranymi urządzeniami podłączonymi do systemu.